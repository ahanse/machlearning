\documentclass[a4paper]{article}

\usepackage[utf8]{inputenc}

\begin{document}
\title{Machine Learning - Exercise 1 }
\author{Deniz Kocabas and Hans-Jörg Schurr}

\maketitle
\tableofcontents
\newpage

Our end report should contain:

\begin{itemize}
	\item Report should be around 10 pages
	\item Full report of your work
	\item Experiments, parameters tried
	\item Characteristics of data sets \& pre-processing (i.e.
handling of missing values, scaling etc.)
	\item Characteristics of regression techniques
    \item Explanation of choice for data sets \& techniques
	\item Discuss experimental results, compare them in regard of
the different datasets \& techniques (tables, figures)
\end{itemize}

\subsection{The chosen datasets}
We had to pick three datasets, which should have different characteristics. 
Our choice was:
\begin{enumerate}
    \item "Individual household electric power consumption Data Set"
        (http://archive.ics.uci.edu/ml/datasets/Individual+household+electric+power+consumption)
    \item "Solar Flare Data Set"
        (http://archive.ics.uci.edu/ml/datasets/Solar+Flare)
    \item "Wine Quality Data Set"
        (http://archive.ics.uci.edu/ml/datasets/Wine+Quality)
\end{enumerate}
The household electric power consumption data set contains over two million
instances and is by far the largest dataset among those three sets. The dataset
has nine dimensions, whereby two fields are used for a time stamp. Thus it is a
timeseries. All other fields contain numeric data. According to the
documentation 1.25 percent of the rows contain missing values. We decided to use
this dataset to test the performance of the methods. Given the high number of
missing values, this dataset is suited to test different methods to handle those
missing values.

The solar flare data set is with roughly 1400 instances much smaller then the
household electric power consumption data set. Every instance has ten fields,
which contain mainly categorical data. Some of those map to numeric intervals
(area of the spots). The set lacks missing values, therefore it is well suited
to test various methods to handle categorical data.

We decided to choose a set with medium size and mixed features as third set. The
wine quality data set provided those requirements. It contains around 5000
instances and no missing values. The twelve dimensions contain mainly numeric
values. Most of those are chemical properties of the wines. The last field is
the an assigned quality of the wine. It is natural number between one and ten.
Given the fact, that it is not exactly measured it may behave like a
categorical value.

\subsection{Methods tried}

\subsection{Experiments and results for the individual datasets}
\subsubsection{The household electric power consumption data set}
\subsubsection{The solar flare data set}
\subsubsection{The wine quality data set}
\subsubsection{Comparison}
\subsection{Conclusion}

\end{document}

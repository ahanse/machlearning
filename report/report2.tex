\documentclass[a4paper]{article}

\usepackage[utf8]{inputenc}
\usepackage{graphicx}
\usepackage{mathtools}

\begin{document}
\title{Machine Learning - Exercise 2 }
%\author{Deniz Kocabas and Hans-Jörg Schurr}
\author{
        Deniz Kucabas \\ 
                     1127055 
                    \and
                    Hans-Jörg Schurr \\ 0925891
}

\maketitle
\tableofcontents
\newpage

\section{Introduction}
The Data Intensive Science has become an indispensable aspect of the peo-
ple's lives. We use it in many different areas to gain knowledge by analysis
of integrated data. The overall goal of that assignment is to extract infor-
mation from that data set and transform it into an understandable structure
for further use. This exercise is about detailed analysis of classification algorithm by using a tool weka.
Nowadays the databases are of huge size,so it is difficult to analize those data,tools like weka are 
devolped. Classification is a data mining (machine learning) technique used 
to predict group membership for data instances. 

We had to pick three datasets from UCI Machine Learning Repository, which should have different characteristics. 
Our choice was:
\begin{enumerate}
    \item "Census-Income (KDD) Data Set" \\ 
(http://http://archive.ics.uci.edu/ml/datasets/Census-Income+%28KDD%29)
    \item "Connect-4 Data Set " \\
        (http://http://archive.ics.uci.edu/ml/datasets/Connect-4)
    \item "ILPD (Indian Liver Patient Dataset) Data Set"\\
        (http://archive.ics.uci.edu/ml/datasets/ILPD+%28Indian+Liver+Patient+Dataset%29)
    \item "Echocardiogram  Data Set" \\
        (http://http://archive.ics.uci.edu/ml/datasets/Echocardiogram)
\end{enumerate}

We have used 5 different classifier algorithm by the tool \emph{Weka}. 
These techniques were;
\begin{enumerate}
    \item Naive Bayes
    \item Decision Tree (J48)
    \item Random Forest
    \item Nearest Neighbor (IBk)
    \item Support Vector Machine (SMO)
\end{enumerate}

\section{The Datasets}
The census income data set contains almost three hundred thousands instances in
two files and is the largest dataset among those three sets. The dataset has
forthy dimensions, whereby some of the fields contain nominal data. There are
two seperate files, one is for testing (1/3 of the data) and one is for
training(2/3 of the data) . We have unified these two files in one. The data set
contain missing values. We decided to use this dataset to test the performance
of the methods. This dataset contains 1 class labels, yearly income. The income
have been binned at the 50000 level to present binary classification. The goal
field of this data, however, was drawn from the "total person income" field
rather than the "adjusted gross income" and may, therefore, behave differently
than the orginal ADULT goal field.

\subsection{Connect-4 data set}
The connect-4 data set is the large dataset, has more than 67000 instances.
Every instance has 42 fields, which contain just categorical data. Some of those
map to numeric intervals
(area of the spots). The set lacks missing values, therefore it is well suited
to test various methods to handle categorical data, with the 1-N coding. The
database has 3 class labels, and a class labels which
is the result of the connect-4 game, i.e. win or loss or draw.

\subsection{ILPD (Indian Liver Patient Dataset) Data Set}
the ILPD data set contains 583 instances with 10 attributes. Every instance represents a
(potential) liver patient. The last attribute is a expert labeled field describing if the patient is 
a liver patient or not. Except of the gender field are all attributes numerical
and with 583 this dataset is a medium size dataset. It has no missing fields. 
We used this dataset to try out the knowledge flow feature of Weka.

\subsection{Ecocardiogram data set}
The last and the smallest dataset is the echocardiogram data set. The
echocardiogram has just 132 instances and also the same number of missing
values. Given the high percentage of
missing values, this dataset is suited to test different methods to handle those
missing values. The twelve attributes contain mainly numeric values. The second
field and the last field can be the class value. All the patients suffered heart
attacks at some point in the past. The first attribute is the number of month
since the patient suffered from the heart attack. 
The second field is whether the patient still
alive or not. The survival and still-alive variables, when taken together,
indicate whether a patient survived for at least one year following the heart
attack (the last attribute). 

\section{Classification Algorithms}

    \subsection{ Naive Bayes}
    \subsection{ Decision Tree (J48)}
    \subsection{ Random Forest}
    \subsection{ Nearest Neighbor (IBk)}
    \subsection{ Support Vector Machine (SMO)}

\section{Experiments and Results for the Individual Datasets}

\subsection{Connect-4 data set}

We have started with the Connect-4 dataset, because the instance number is in the middle and it is suitable to find the best parameter sets of the algorithms to use them for the next datasets.

Connect-4 is a two player game and normally the players choose a color disk, but in our dataset, first player uses 'x' which is called player x and second player, i.e. player o uses 'o'. Then the players start to drop their pieces into a seven column and six row vertically suspended grid. The disks fall straight down, occupying the next available space within column. The goal of the game is to connect four of one's own discs of the same letter next to each other vertically, horizontally, or diagonally before your opponent. ($http://en.wikipedia.org/wiki/Connect_Four$) 

The class variable can be win or loss or draw. The distribution of it is; 44473 records win, i.e. 65.83 $\%$, 16635 records loss,  i.e. 24.62 $\%$ and 6449 records draw,  i.e. 9.55 $\%$.

The game has played 8 turns and none of the player has won yet. The attributes are all of the square in the seven column and six row board, which can be $\{$x, o, b$\}$ i.e. x is for the player x, o is for player o and b for blank. So, the dataset includes 42 attributes for the board grid. The data set has not got any missing values. Every instances has 4 times x, 4 times o values and the rest are b. 

We have chosen this data set due to test preprocess for 1-N coding. The function for one code encoding in weka is NominalToBinary. And after we have run that function all of these columns changed to the column$\_$name=x, column$\_$name=o, column$\_$name=b, one of it will be 1 and the others will be 0, depends on a value of the column$\_$name. After the preprocess there are 127 attributes with the class label. 

After we have finished preprocess steps, we have run one by one NaiveBayes, Decision Tree (J48), Random Forest, Nearest Neighbor (IBk), Support Vector Machine (SMO) Classification algorithm, which are sufficiently different. The most important value in the results of the classification algorithms are the correctly and incorrectly classified instance percentage. How more the correctly instance is, the classification algorithm has more performance. These percentages are calculated from the confusion matrix. Add all the correct values of that matrix and add all the incorrect values of that matrix will give the percentage of that values. 

At first we have started with the Naive Bayes techniques. In that classification the algorithm has 4 parameters, which are debug, useSupervisedDiscretization, useKernelEstimator, displayModelInOldFormat. useKernelEstimator and useSupervisedDiscretization parameters are for numeric attributes, so that we dont need to use them. if the debug parameter is true, classifier may output additional info to the console. In displayModelInOldFormat parameter we have one class and many attributes, so that the new format is better to use in our dataset. In consequence, we set all the parameters false like in the default version and use the algorithm in 10-fold cross validation. We run the algorithm two times to see the effect of the preprocess strategy, which is one code encoding. The result of the Naive Bayes is below in Table 1.

\begin{table}
\begin{tabular}{|l| c | c | c |c |c |}

\hline & & & & & \\
Pre- & Cor. Classified . & Tra. & Val. & Mean abs.  & Root mean \\
Process & Instance & T(s) &  T(s) & Error & Squared Error \\
\hline & & & & & \\
- &72.14  $\%$ & 0.08 & 3 &  0.2672 & 0.3587 \\ 
\hline & & & & & \\
1-N c. &64.23  $\%$ & 4.38 & 103 & 0.2783 & 0.4129 \\ 
\hline
\end{tabular}
\caption{Results of the Naive Bayes Classification Algorithms}
\end{table}

Secondly, i will explain the J48 classification i.e. one of the decision tree algorithm. In that classification the algorithm has several parameters. I will pass the parameters that i explained above. minNumObj parameter set the minimum number of instances per leaf.(Default: 2) confidenceFactor sets confidence threshold pruning, the default value is 0.25 and the smaller values incur more pruning. binarySplits has been used on nominal attributes when building the trees. seed parameter used for randomizing the data when reduced error pruning. numFolds parameter determines the amount of data used for reduced error pruning.  One fold of numFolds is used for pruning, the rest for growing the tree. (Default: 3) unpruned parameter is to prunning performed or not. saveInstanceData is to save the training data for visualization. subtreeRaising don't perform subtree raising. collapseTree is for parts are removed that do not reduce training error. useMDLcorrection is for finding splits on numeric attributes. useLaplace counts at leaves are smoothed based on Laplace. reducedErrorPruning reduced-error pruning is used instead of C.4.5 pruning. 

I have changed some of the parameters of J48. When i increase the confidenceFactor, i achieved more leaves and the size of tree extends and it took more seconds, but had better correctly classified instances. After that i increase the number of instances per leaf with the parameter minNumObj, but the result was worser than the default value. i set subtreeRaising parameter false, the result was also not good and i change these to the default values. All of the result of J48 classification is in Table 1. Finally i have had the best result of the J48 algorithm with these instances J48 -C 0.5 -M 2, there is just one difference from the default values, i.e confidenceFactor = 0.5 instead of 0.25. (Table 2) 

\begin{table}
\begin{tabular}{|l| c | c | c | c |c |c |}

\hline & & & & & & \\
Parameters & Pre- & Cor. C.& Tra. & Val. & Nr. of  &  Tree  \\
 & process  & Instance & T(s) &  T(s) & Leaves & Size \\
\hline & & & & & & \\
J48 -C 0.25 -M 2 	 & - &			80.97  $\%$ & 6.66 & 85 & 4297  & 6445 \\ 
\hline & & & & & & \\
J48 -C 0.25 -M 2 	& 1-N c. &	80.06 $\%$ & 254.09 & 2755 & 3495  & 6989  \\ 
\hline & & & & & & \\
J48 -C 0.1 -M 2 	 & - &			80.15  $\%$ & 6.62 & 79 &  2217 & 3325\\ 
\hline & & & & & & \\
J48 -C 0.5 -M 2 	& - &			80.98 $\%$ & 2.63 & 79 &  6543 & 9814 \\ 
\hline & & & & & & \\
J48 -C 0.5 -M 5  	 & - &			80.49 $\%$ & 2.52 & 85 &  3403 & 5104 \\ 
\hline & & & & & & \\
J48 -S -C 0.5 -M 2 	& -&			 80.83 $\%$ & 5.36 &  73 & 7079 & 10618\\ 
\hline
\end{tabular}
\caption{Results of the J48 Classification Algorithms with different Parameters}
\end{table}

The third algorithm is the Randomforest algorithm. The technique has several parameters. I will pass again the parameters that i explained above. printTrees parameter is to print the individual trees in the output. So, we do not need to show it and left it in the false for this dataset. Because there will be many leaf and the trees are quiet huge to show here. maxDepth is the maximum depth of the trees. We have chosen the default value 0. unlimited.numExecutionSlots is the number of execution slots to use for constructing the ensemble. The randomforest algorithm, which builds a series of trees, so the parameter numTrees is to how many trees to be generated. The default value is 10, we have changed it to see the effect of the tree numbers. numFeatures is the number of attributes to be used in random selection. 

After we have run the RandomForest algorithm, for the different number of trees in the forest. If we increase the number of trees, the algorithms runtime would be slower, but also the results will be better. So the result is improved for more tree values. The results of the  Randomforest  strategy for the different number of trees are in the Table 3.


\begin{table}
\begin{tabular}{|l| c | c | c | c |c |c |}

\hline & & & & & \\
Nr. Of & Pre. & Cor. C.& Tra. & Val. & Out of Bag  \\
Trees & Process & Instance & T(s) &  T(s) & Error  \\
\hline & & & & & \\
10	 & - &			79.93  $\%$ & 12.39 & 168 & 0.2425  \\ 
\hline & & & & & \\
10 	 & 1-N c. &	79.93  $\%$ & 12.39 & 168 & 0.2425  \\ 
\hline & & & & & \\
5 	& - &			77.69 $\%$ & 7.51 & 81 & 0.2796   \\ 
\hline & & & & &  \\
50 	 & - &			82.19  $\%$ & 65.24 & 753 &  0.1868 \\ 
\hline
\end{tabular}
\caption{Results of the RandomForest Classification Algorithms with different Parameters}
\end{table}

The fourth strategy is the  Nearest Neighbor algorithm, in Weka IBk. The technique has also several parameters. We have tested these algorithm for the KNN parameter ,which is most important parameter in that algorithm. The default value of it is 1. The nearestNeighbourSearchAlgorithm parameter is the nearest neighbour search algorithm to use. The distance metric can be simple Euclidean distance, which is also the default value. The results of the  Nearest Neighbor  strategy for different k parameter is in the Table 4.

\begin{table}
\begin{tabular}{|l| c | c | c | c |c |c |}

\hline & & & & & & \\
k &  Pre- & Cor. C. & Tra. & Val. & Mean abs.  & Root mean \\
Parameter& Process & Instance & T(s) &  T(s) & Error & Sq. Error \\
\hline & & & & & & \\
1 	 & - &			80.9213		   $\%$ &	0.04		& 	3090	 & 	0.2243	 & 0.307  \\ 
\hline & & & & & & \\	
1 	 & 1-N c.  &				   $\%$ &			& 		 & 		 &  \\ 
\hline & & & & & & \\
5	 & - &		  			 $\%$ &			& 		 & 		 &  \\ 
\hline & & & & & & \\	
10	 & - &		  			 $\%$ &			& 		 & 		 &  \\ 
\hline
\end{tabular}
\caption{Results of the IBk Classification Algorithms with different Parameters}
\end{table}

The last classification technique that we have used is the Support Vector Machine, in Weka we have used SMO algorithm. The first -C is complexity parameter, which SMO support vector machine uses to
build the hyperplane between any two target classes. A support vector machine constructs a hyperplane or set of hyperplanes in a high or infinite dimensional space. Intuitively, a good separation is achieved by the
hyperplane that has the largest distance to the nearest training data points of any class, since in general the larger the
margin the lower the generalization error of the classifier. So -C controls how soft the class margins are, in
practice how many instances are used as 'support vectors' to draw the linear separation boundary in the transformed euclidean feature space. Gamma parameter is also critical for performance but obsolete in the linear kernel. (Default : PolyKernel -E 1.0) . The results of the  SVM  strategy for different k parameter is in the Table 5.


\begin{table}
\begin{tabular}{|l| c | c | c | c |c |c |}

\hline & & & & & & \\
Complexity & Pre- &  Cor. C. & Tra. & Val. & Mean abs.  & Root mean \\
Parameter & Process & Instance & T(s) &  T(s) & Error & Sq, Error \\
\hline & & & & & & \\
1.0 & - &		   $\%$ &			& 		 & 		 &  \\ 
\hline & & & & & & \\
1.0  & 1-N c. &	   $\%$ &			& 		 & 		  &  \\ 
\hline & & & & & & \\
0.1	& - &			  $\%$ & 		& 		 & 		  &   \\ 
\hline & & & & & & \\
10	 & - &			   $\%$ &   		& 		 &  		 & \\ 
\hline
\end{tabular}
\caption{Results of the Support Vector Machine Classification Algorithms with different Parameters}
\end{table}


\subsection{Census-Income (KDD) data set}

The census income data set has a huge instance, which has almost three hundred thousands  instances. First we convert the file to csv and try to upload it to \emph{Weka}. At first weka didnt notice the missing values and add a new attribute value with "?". Then we have deleted the space before question mark and the problem is solved. The maxheap value was 4gb, so we had not any problem to upload instance file to weka. 

The class variable yearly income has two possible value, that are more and less than 50k \$. More than 50k \$ is 280717 records and the less than 50k \$ is 18568. 

Some of the attributes have half percent of missing values, we need to deal with these missing values in preprocess segment. There is a function, which is called Replacemissingvalues in Weka. This function replaces all missing values for nominal and numeric attributes in dataset with the modes and mean from the training data. In that dataset, there are also many nominal attributes.. After run the Replacemissingvalues function, we have passed the section to classify of the \emph{Weka}. 












\subsection{Echocardiogram data set}

The Echocardiogram data set is the smallest dataset, which has around 132 instances. First we convert the file into excel file and upload it to \emph{Weka}. In this file, there are missing values, the same number of the instances. So we do need also to run Replacemissingvalues in \emph{Weka} preprocess tab. The most of the missing values are in the last column, which can also be the class attribute. The 58 of the values are missing and it is almost half of the instances. In that reason, we have chosen the still alive column as a class attribute instead of alive at 1, that indicate whether a patient survived for at least one year following the heart attack. 

After we have finished preprocess steps, we have run the 5 classification algorithm and have used also the 10 folds Cross-validation techniques to split the data for test and train. 

At first we have started with the Naive Bayes techniques,







\subsection{Comparison}

\section{Conclusion}

\end{document}
